\documentclass{jarticle}

\setlength{\textwidth}{185mm}
\setlength{\textheight}{225mm}
\setlength{\topmargin}{0cm}
\setlength{\oddsidemargin}{-1cm}
\setlength{\evensidemargin}{-1cm}

\usepackage[dvipdfmx]{graphicx} %PDF用
\usepackage{eclbkbox} %breakbox用


\title{プログラム入門 レポート}
\author{145770F 秋田 海人}
\date{}

\begin{document}
\maketitle
\parindent = 0pt


\section{第 I 章\\}

\subsection{問1-1\\}
\subsubsection{プログラム\\}
\begin{breakbox}
\begin{verbatim}
#include <stdio.h>

int main(void) {
	printf("Hello \n World!");
}
\end{verbatim}
\end{breakbox}
\subsubsection{実行結果\\}
\begin{breakbox}
\begin{verbatim}
Hello 
 World!
\end{verbatim}
\end{breakbox}
\subsubsection{考察\\}

\subsection{問1-2\\}
\begin{breakbox}
\begin{verbatim}

\end{verbatim}
\end{breakbox}
\subsubsection{考察\\}

\subsection{問1-3\\}
\subsubsection{プログラム\\}
\begin{breakbox}
\begin{verbatim}

\end{verbatim}
\end{breakbox}
\subsubsection{考察\\}


\subsection{問1-4\\}
\subsubsection{プログラム\\}
\begin{breakbox}
\begin{verbatim}
#include <stdio.h>

int main(void) {
	int ax, ay, az;
	int bx, by, bz;
	int n, g_1, g_2, g_3;
  printf("Input ax, ay, az \n");
	scanf("%d %d %d", &ax ,&ay, &az);
  printf("Input bx, by, bz \n");
	scanf("%d %d %d", &bx ,&by, &bz);
  
	n = (ax * bx) + (ay * by) + (az * bz); 
	g_1 = (ay * bz) - (az * by); 
	g_2 = (az * bx) - (ax * bz); 
	g_3 = (ax * by) - (ay * bx);

	printf("内積: %d \n", n);
	printf("外積: (%d, %d, %d)\n", g_1, g_2, g_3); 
}
\end{verbatim}
\end{breakbox}
\subsubsection{実行結果\\}
\begin{breakbox}
\begin{verbatim}
Input ax, ay, az 
1 0 1
Input bx, by, bz 
1 1 0
内積: 1 
外積: (-1, 1, 1)
\end{verbatim}
\end{breakbox}
\subsubsection{考察\\}


\subsection{問1-5\\}
\subsubsection{プログラム\\}
\begin{breakbox}
\begin{verbatim}
#include <stdio.h>

int main(void) {
	double ax, ay, az;
	double bx, by, bz;
	double n, g_1, g_2, g_3;
  printf("Input ax, ay, az \n");
	scanf("%lf %lf %lf", &ax ,&ay, &az);
  printf("Input bx, by, bz \n");
	scanf("%lf %lf %lf", &bx ,&by, &bz);
  
	n = (ax * bx) + (ay * by) + (az * bz); 
	g_1 = (ay * bz) - (az * by); 
	g_2 = (az * bx) - (ax * bz); 
	g_3 = (ax * by) - (ay * bx);

	printf("内積: %lf \n", n);
	printf("外積: (%lf, %lf, %lf)\n", g_1, g_2, g_3); 
}
\end{verbatim}
\end{breakbox}
\subsubsection{実行結果\\}
\begin{breakbox}
\begin{verbatim}
Input ax, ay, az 
1.01 0 1.12
Input bx, by, bz 
1.11 1.32 0
内積: 1.121100 
外積: (-1.478400, 1.243200, 1.333200)
\end{verbatim}
\end{breakbox}
\subsubsection{考察\\}

\subsection{問1-6\\}
\subsubsection{プログラム\\}
\begin{breakbox}
\begin{verbatim}
#include <stdio.h>
int main(void)
{
    double x, dx, dfdx_num;
    x = 1.0;
    
    printf("Input delta x .\n delta x = ");
    scanf("%lf", &dx);
    
    dfdx_num = ((x+dx) * (x+dx) * (x+dx) - (x*x*x)) / dx;
    
    printf("df/dx(x=1) = 3. delta x = %lf\n", dx);
    printf("Numerical value of df/dx = %lf\n", dfdx_num);

}
\end{verbatim}
\end{breakbox}

\subsubsection{実行結果\\}
\begin{breakbox}
\begin{verbatim}
Input delta x. 
 delta x = 0.1 
 df/dx(x=1) = 3. delta x = 0.100000 
Numerical value of df/dx =  3.310000 
\end{verbatim}
\end{breakbox}

\begin{breakbox}
\begin{verbatim}
Input delta x. 
 delta x = 0.01
 df/dx(x=1) = 3. delta x = 0.010000 
Numerical value of df/dx =  3.030100 
\end{verbatim}
\end{breakbox}

\begin{breakbox}
\begin{verbatim}
Input delta x. 
 delta x = 0.001
 df/dx(x=1) = 3. delta x = 0.001000 
Numerical value of df/dx =  3.003001 
\end{verbatim}
\end{breakbox}

\begin{breakbox}
\begin{verbatim}
IInput delta x. 
 delta x = 0.0001
 df/dx(x=1) = 3. delta x = 0.000100 
Numerical value of df/dx =  3.000300 
\end{verbatim}
\end{breakbox}
\subsubsection{考察\\}

\subsection{問1-7\\}
\subsubsection{プログラム\\}
\begin{breakbox}
\begin{verbatim}
#include <stdio.h>
int main(void)
{
    double x, dx, dfdx_num;
    x = 1.0;
    
    printf("Input delta x =");
    scanf("%lf", &dx);
    
    dfdx_num = ((x+dx) * (x+dx) * (x+dx) - (x*x*x)) / dx;
    
    printf("df/dx(x=1) = 3. delta x = %lf\n", dx);
    printf("Numerical value of df/dx = %lf\n", dfdx_num);
    printf("Relative error |3.0 - df/dx| / 3.0 = %lf\n", ((3.0-dfdx_num)/3.0));
}
\end{verbatim}
\end{breakbox}

\subsubsection{実行結果\\}
\begin{breakbox}
\begin{verbatim}
Input delta x. 
 delta x = 0.1 
 df/dx(x=1) = 3. delta x = 0.100000 
Numerical value of df/dx =  3.310000 
Relative error |3.0 - df/dx| / 3.0 = 0.103333 
\end{verbatim}
\end{breakbox}

\begin{breakbox}
\begin{verbatim}
Input delta x. 
delta x = 0.01
df/dx(x=1) = 3. delta x = 0.010000 
Numerical value of df/dx =  3.030100 
Relative error |3.0 - df/dx| / 3.0 = 0.010033 
\end{verbatim}
\end{breakbox}

\begin{breakbox}
\begin{verbatim}
Input delta x. 
 delta x = 0.001 
 df/dx(x=1) = 3. delta x = 0.001000 
Numerical value of df/dx =  3.003001 
Relative error |3.0 - df/dx| / 3.0 = 0.001000 
\end{verbatim}
\end{breakbox}

\begin{breakbox}
\begin{verbatim}
Input delta x. 
 delta x = 0.0001 
 df/dx(x=1) = 3. delta x = 0.000100 
Numerical value of df/dx =  3.000300 
Relative error |3.0 - df/dx| / 3.0 = 0.000100 
\end{verbatim}
\end{breakbox}
\subsubsection{考察\\}

\subsection{問1-8\\}
\subsubsection{プログラム\\}
\begin{breakbox}
\begin{verbatim}
#include<stdio.h>
#include <math.h>

int main(void){
	double x, dx, dfdx_num;
	x = 1.0;
	printf("Input delta x. \n delta x = ");
	scanf( "%lf", &dx);
  	dfdx_num = (pow((x+dx),3.0) - pow((x-dx), 3.0)) / (2*dx);	
	printf("df/dx(x=1) = 3. delta x = %lf \n", dx);
	printf("Numerical value of df/dx = % lf \n", dfdx_num);
  	printf("Relative error |3.0 - df/dx| / 3.0 = %lf \n", fabs(3.0 - dfdx_num) / 3.0);
}
\end{verbatim}
\end{breakbox}
\subsubsection{実行結果\\}
\begin{breakbox}
\begin{verbatim}
Input delta x. 
 delta x = 0.1
 df/dx(x=1) = 3. delta x = 0.100000 
Numerical value of df/dx =  3.010000 
Relative error |3.0 - df/dx| / 3.0 = 0.003333 
\end{verbatim}
\end{breakbox}

\begin{breakbox}
\begin{verbatim}
Input delta x. 
 delta x = 0.01
 df/dx(x=1) = 3. delta x = 0.010000 
Numerical value of df/dx =  3.000100 
Relative error |3.0 - df/dx| / 3.0 = 0.000033 
\end{verbatim}
\end{breakbox}

\begin{breakbox}
\begin{verbatim}
Input delta x. 
 delta x = 0.001
 df/dx(x=1) = 3. delta x = 0.001000 
Numerical value of df/dx =  3.000001 
Relative error |3.0 - df/dx| / 3.0 = 0.000000 
\end{verbatim}
\end{breakbox}

\begin{breakbox}
\begin{verbatim}
Input delta x. 
 delta x = 0.0001
 df/dx(x=1) = 3. delta x = 0.000100 
Numerical value of df/dx =  3.000000 
Relative error |3.0 - df/dx| / 3.0 = 0.000000 
\end{verbatim}
\end{breakbox}
\subsubsection{考察\\}

\subsection{問1-9\\}
\subsubsection{プログラム\\}
\begin{breakbox}
\begin{verbatim}
#include<stdio.h>
#include<math.h>

#define  N 6

int main(void){

    double x, z, fx, f[N];
    int i, j;
    x = 1.0;
    z = 1.0;
    fx = exp(x);
    f[0] = 1.0;

    for(i = 1; i < 6; i++ ){
        z *= i;

        f[i] = f[i-1] + pow(x, i) / z;
    }

    for(j = 0; j < 6; j++){
        printf("exp(%lf) = %lf, %d order = %lf, error = %lf \n", x, fx, j, f[j], fabs((fx - f[j])/fx));
    }

}
\end{verbatim}
\end{breakbox}

\subsubsection{実行結果\\}
\begin{breakbox}
\begin{verbatim}
exp(1.000000) = 2.718282, 0 order = 1.000000, error = 0.632121 
exp(1.000000) = 2.718282, 1 order = 2.000000, error = 0.264241 
exp(1.000000) = 2.718282, 2 order = 2.500000, error = 0.080301 
exp(1.000000) = 2.718282, 3 order = 2.666667, error = 0.018988 
exp(1.000000) = 2.718282, 4 order = 2.708333, error = 0.003660 
exp(1.000000) = 2.718282, 5 order = 2.716667, error = 0.000594 
\end{verbatim}
\end{breakbox}
\subsubsection{考察\\}

\section{第 II 章\\}

\subsection{問2-1\\}
\subsubsection{プログラム\\}
\begin{breakbox}
\begin{verbatim}
#include <stdio.h>

int main(void){
	int i, n, sum1, sum2;

	printf("n:");
	scanf("%d",&n);

	sum1 = 1;
	sum2 = 1;

	for(i=0; i<n; i++){
		printf("i = %3d, sum1 = %3d\n", i, sum1);
		sum1 = sum1 + 1;
	}

	printf("nizyou\n");
	for(i=0; i<n; i++){
		sum2 += i * i;
		printf("i = %3d, sum2 = %3d\n", i, sum2);
	}

}
\end{verbatim}
\end{breakbox}
\subsubsection{実行結果\\}

\begin{breakbox}
\begin{verbatim}
\end{verbatim}
\end{breakbox}


\begin{breakbox}
\begin{verbatim}
\end{verbatim}
\end{breakbox}



\begin{breakbox}
\begin{verbatim}
\end{verbatim}
\end{breakbox}

\begin{breakbox}
\begin{verbatim}
\end{verbatim}
\end{breakbox}

\begin{breakbox}
\begin{verbatim}
\end{verbatim}
\end{breakbox}
\begin{breakbox}
\begin{verbatim}
\end{verbatim}
\end{breakbox}

\end{document}
